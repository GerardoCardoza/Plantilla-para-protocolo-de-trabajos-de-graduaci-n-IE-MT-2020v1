

Este proyecto consta en la simulación y verificación de los archivos obtenido del diseño lógico y la extracción de parásitos de los componentes del circuito realizado. 
En la parte del diseño lógico se enfoca mas en lo que es la depuración y optimización en el flujo de diseño con tecnología de 180nm, en donde esto consta de un flujo de diseño para poder fabricar el chip. Esta parte se va a enfocar en optimizar para la prueba y simulación de archivos HDL y esquemáticos que se pueden llegar a generar las herramientas de synopsys, en esta parte del diseño lógico se utilizara tanto como VCS y Verdi.
En esta parte se realizaran distintas pruebas, para que en la parte final, ya se pueda validar cada uno de los flujos de diseño.

En la parte de parásitos se va a encargar de la simulación en HSpice, que nos permite realizar una extracción exitosa para realizar el chip y lograr ver si cumple con las especificaciones. Se pretende mostrar lo que realmente hace el software, las librerías y los comandos que se vana realizar en este proyecto, para que las personas en un futuro puedan agarrar el flujo y realizar sus propios proyectos.
