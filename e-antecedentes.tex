

Este proyecto se puedo realizar gracias al Ingeniero Carlos Esquit, que fue asignado en el 2009 como el director de carrera del departamento de Ingeniería de Electrónica y Mecatronica. Esta persona fue que comenzó con esta iniciativa por el 2013.

Todo comenzó cuando comenzaron a dar el curso de VLSI en la Universidad del Valle de Guatemala, en el que paso a ser Nanoelectronica 1, en donde se aprendieron todos los conceptos necesarios para realizar esta parte del proyecto. 

Cabe resaltar que todo esto se pudo realizar gracias al acuerdo que se hizo con Synopsys, para que la mayoría de las partes involucradas se pudieran realizar desde la casa, por los problemas de la pandemia que dieron lugar a comienzos del 2020.

En el 2019 un grupo de estudiantes de la Universidad logro comenzar a realizar un proyecto de un chip a escala nanométrica, desde que se comenzó con esta iniciativa otro grupo de estudiantes en el 2020 comenzaron a realizar retoques en este proceso de poder realizar un chip a nanoescala.  \cite{hoover2010bio} \cite{park2014design}